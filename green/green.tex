%\documentclass[aps,twocolumn,superscriptaddress,pra]{revtex4}
\documentclass[aps,onecolumn,superscriptaddress,showpacs,pra]{revtex4}
%%%%%%%%%%%%%%%%%%%%%%%%%%%%%%%%%%%%%%%%%%%%%%%%%%%%%%%%%%%%%%%%%%%%%%%%
\usepackage{graphicx,amsmath,color}
\usepackage{epstopdf}
\usepackage{bm}
\usepackage{amsfonts}

\begin{document}

    \title{The equation of motion in dumped oscillation explanation for the Green function}

    \author{Tsuyoshi Kadokura}
    \affiliation{University of Electro-Communications}

    \date{\today}
    \begin{abstract}

        We wourd like to expeHello World! \ 
	    \begin{displaymath}
            \frac{\partial v}{\partial t}
        \end{displaymath}
    \end{abstract}

    %\pacs{xx.xx.Mn, xx.xx.Lm, xx.xx.Bc, xx.xx.Fg}

    \maketitle

    \section{INTRODUCTION}
    \label{s:introduction}

    The successful experimental realization of various synthetic gauge fields
    drawn considerable theoretical attention~\cite{A. Messiah}
    with a number of studies.


    The remainder of the present paper is organized as follows. In
    Sec.~\ref{s:formulation}, we formulate the theoretical model
    for a random potential obey the Langevin equation.
    The parameter dependence and velocity field are investigated in
    Secs.~\ref{s:dynamics} and \ref{s:VF}, respectively.
    Finally, in Sec.~\ref{s:conclusions}, the main results of the present
    paper are summarized.

    \section{Formulation of the problem}
    \label{s:formulation}

    We consider a isotropic energy boundaries.
    The approximation of the many-body effects is used, and the dynamics of the system is
    describied by the GP equation as follows:

    \begin{eqnarray}
        \label{eq:GP}
        i \hbar \frac{\partial}{\partial t}\bm \Psi =
        - \frac{\hbar^2}{2m}\bm{\nabla}^2\bm{\Psi}
        &+&i\frac{\hbar k_0}{m} \bm{\nabla} \cdot \bm{\sigma}_\perp \bm\Psi
        \nonumber\\
        &+&U(\bm r,t)\bm\Psi
        + \hat{G}\left( \bm\Psi,\bm \Psi^\dagger\right)\bm\Psi,
    \end{eqnarray}


    %\section{Dynamics of the system}
    %\label{s:dynamics}

    %consideration qualitative


    %\section{Parameter dependence of the critical velocity}
    %\label{s:parameter}

    %consideration quantitative


    %\section{Conclusions}
    %\label{s:conclusions}

    %In conclusion, we have investigated the dynamics of

    %\begin{acknowledgments}
    %    The present study was supported by JSPS KAKENHI Grant Numbers %JPxxxxxxxx,
    %\end{acknowledgments}

    \begin{thebibliography}{99}
%%%%%%%%%%%% Galilean transformation %%%%%%%%%%%%%%%%%%%%%%%%%%%%%%%%
        \bibitem{A. Messiah}
        A. Messiah, {\it Quantum Mechanics} (North-Holland, Amsterdam, 1961).

        \bibitem{SM}
        See Supplemental Material at http:// for movies of the dynamics.

    \end{thebibliography}

\end{document}
%   eof
