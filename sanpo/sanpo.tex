%
%	filename:	sanpo.tex
%	contents:	道端ノート
%	author:		1643002:門倉 強
%
\documentclass[12pt,a4paper]{jbook}
%\documentclass[12pt,a4paper]{jarticle}
%\documentclass[10pt,a4paper]{book}
%\documentclass[10pt,a4paper]{report}
\setlength{\oddsidemargin}{0mm}
\setlength{\evensidemargin}{0mm}
\setlength{\topmargin}{0mm}
\setlength{\headsep}{2truemm}
\setlength{\textwidth}{460pt}
\setlength{\textheight}{25.00cm} %{22.80cm}
\setcounter{tocdepth}{2}
\usepackage{amsmath,amssymb,bm}
\usepackage[dvipdfmx]{graphicx}
\usepackage{cases}
\usepackage{ascmac}
%\usepackage{wasysym}
\usepackage{multicol}
\usepackage{mathrsfs}
\usepackage[scriptsize]{caption} % 2014/09/27
\usepackage{fancyhdr}
%\usepackage[dvipdfmx]{hyperref}
%\usepackage{showkeys}
\newcommand{\diff}{\mathrm{d}}				           %
\newcommand{\divergence}{\mathrm{div}\,}		       %
\newcommand{\grad}{\mathrm{grad}\,}			           %
\newcommand{\rot}{\mathrm{rot}\,}			           %
\newcommand{\const} {{\rm const.}}			           %
\newcommand{\h}  {{\rm h}}				               % プランク定数
\newcommand{\na} {N_{\rm A}}				           % アボガドロ定数
\newcommand{\kb} {k_{\rm B}}				           %
\newcommand{\me} {m_{\rm e}}				           %
\newcommand{\lap} {\bigtriangleup}			           %
\newcommand{\dal} {\Box}				               %
\renewcommand{\figurename}{FIG.}			           %
\def\Vec#1{\mbox{\boldmath $#1$}}			           %
\def\fourier#1 { {\cal F} \left\{ #1 \right\} }		   %
\def\ifourier#1 { {\cal F}^{-1} \left\{ #1 \right\} }  %
\def\laplace#1 { {\cal L} \left\{ #1 \right\} }		   %
\def\ilaplace#1 { {\cal L}^{-1} \left\{ #1 \right\} }  %
\title{
{\Huge \tt
道端ノート} \\
%{\Large 2021 The University of Electro-Communications \\
%Doctor's thesis} \\[1cm]
}
\author{
%\LARGE 学籍番号: \fbox{1643002} \\[0.5cm]
%\LARGE 氏名: \fbox{門倉 強} \\[0.5cm]
%\LARGE 門倉 強 \\[0.5cm]
\\
\large Tsuyoshi Kadokura \\[0.5cm]
%\LARGE SAITO Research Laboratory \\[0.5cm]
\date {}
}
\begin{document}
\maketitle
\makeatletter
\def\ps@fancy{%
\def\chaptermark##1{\markboth{\ifnum \c@secnumdepth>\z@ \thechapter\hskip 0.5em\relax \fi ##1}{}}%
\def\sectionmark##1{\markright {\ifnum \c@secnumdepth >\@ne \thesection\hskip 0.5em\relax \fi ##1}}%
\ps@@fancy
\gdef\ps@fancy{\@fancyplainfalse\ps@@fancy}%
\ifdim\headwidth<0sp
\global\advance\headwidth123456789sp\global\advance\headwidth\textwidth
\fi
}
\makeatother
\normalsize
\pagestyle{fancy}
\tableofcontents
\setcounter{page}{1}
\pagenumbering{arabic}
\chapter{お散歩}
\section{歩く音、遠くの音}
\section{ Near-field, Far-field}
\section{局所慣性系}
・散歩の愉しみ、手元の道端ノートはキーワードのマルチツール、
\\
・広く浅くから狭く深く、研究室は推敲の道
\\
・事故は一瞬、悔いは一生
\\
・酒は引き算、断酒は足し算、
\\
・堰の流れ、杭の渦
\section{Sメーターの灯り}
・トランジスタラジオ
ミツミ三端子ラジオ+チューニングメーター
\section{多摩地域広播電台}
1ch : TBSラジオ
    90.5MHz
\\
2ch : 文化放送
    91.6MHz
\\
3ch : ニッポン放送
    93.0MHz
\\
4ch : NHK-FM
    82.5MHz
\\
5ch : 調布FM
    83.8MHz(調布花火大会)
\\
6ch : 東京FM
    80.0MHz
\\
月曜日:壇蜜
金曜日:SHIBA-HAMA
土曜日:徳光和夫、いとうあさこ、菊池桃子、吉田照美、伊東四朗、みむこじ、ほっこり、
日曜日:、薬師丸ひろ子、三宅裕司、土田晃之、爆笑問題、鶴瓶、
\section{ハム/ポータブルワン}
・調布アマチュア無線クラブ
 JA1   438.36MHz
・電気通信大学無線部
 JA1ZGP
\section{Air Band}
調布飛行場
\section{枕木線路のハルジオン}
・京王線
 調布駅:電気通信大学、新光書店、マック、千年ラーメン、ダイソー、布田天神社、ゲゲゲの鬼太郎
 布田駅:千の藤、
 西調布駅:江川亭、セリア、100円ローソン(ベーコン)、オザム、
 飛田給駅:
 京王多摩川駅:かたやきそば、多摩川児童公園、朝鮮人古住居、北川電機、つげ義春
 稲城駅:図書館、三沢川、富士そば、
 7000形、8000形、9000形、5000形ライナー
\\
・南武線
 武蔵溝ノ口駅:いろは商店街、
 稲田堤駅:たぬきや、松屋夜まつ
 矢野口駅:すきや、
 稲城長沼駅:富士プラ、三来軒 中華丼、ガンダム、シャーザク、中村そば、
 南大丸駅:ヤッターワン、
 立川駅:カニチャーハン、焼きそば、屋台村、おでんそば、下新田ノ津島神社
\\
・西武多摩川線
 是政駅:にんにくラーメン、
 競艇場前駅:ローソンの肉まん、
 多磨駅:もも屋うどん、探鳥会、
 赤電、黄電、青電、伊豆箱根鉄道、白電
\\
・中央線
 武蔵境駅:餃子の満州、
 国分寺駅:本屋、文房具屋、
\\
・小田急線
 登戸駅:
 海老名駅:箱根そば、
\section{受令波}
救急・消防
調布、稲城、府中、狛江、三鷹、府中、国分寺、多摩市
\section{青空ごはん}
・食堂
 A・B・C定 360円
 うどん・そば 230円
・折り畳み椅子
・日よけ帽子
・ポケットカメラ
・落花生、パン、ALL FREE
・ポケットストーブ、燃料、飯盒、手袋、米 150cc、水 180cc
・弁当
・自転車、論文、紙、ラジオ、道端ノート
\section{道端の植物園}
・マメグンバイナズナ
・ハルシオン
・カタバミ
・タネツケバナ
・ユウゲショウ
・ヒメツルソバ
・オオバコ
・エノコロクサ
・チガヤ
・セイタカアワダチソウ
・マツヨイグサ
\section{探鳥会}
・スズメ
・ドバト
・キジバト
・ハシブトガラス
・ハシボソガラス
・ムクドリ
・オナガ
・セグロセキレイ
・ツバメ
・メジロ
・ガビチョウ
・チョウゲンボウ
・アイガモ
・シロサギ
・カワセミ
・
\section{地図}
・稲田堤、矢野口、稲城、稲城長沼、是政、小柳、多摩川、調布、調布ヶ丘、上石原、下石原
\\
\includegraphics[width=250mm,angle=90]{inadadutsumi.eps}
\\
\includegraphics[width=250mm,angle=90]{inaginaganuma.eps}
\\
\includegraphics[width=250mm,angle=90]{koremasa.eps}
\chapter{道端の文学}
・端緒
\\
えんぴつ、温泉、線路、屋台、酒場、トタン、貨車、道端、立ち飲み、
三角、微分、ラジオ、放送、受信、海外、飛行機、プロペラ、タイヤ、
荷台、リヤカー、手押し車、側溝、砂利道、自転車、管制塔、散歩道、
尾灯、電灯、ひさし、雨宿り、軒先、縁側、縁の下、冷凍庫、氷、
電線、電柱、碍子、空中線、ローソク、雷、地蔵、雨音、靴音、
イヤホン、レシーバ、計算尺、曲がり角、蛍光灯、ハム音、団地、
計算機、電鍵、灯り、取っ手、ドアノブ、真鍮、蝶番、お勝手、
風呂、薪、ストーブ、やかん、蛇口、ベローズ、障子、日陰、行灯、
野良猫、パンク、同調、呼び鈴、庚申塔、切符、電話、保線区、
角運動量、工場、タバコ、八百屋、紙芝居、飴玉、水たまり、
橋、歩道、手すり、階段、競輪場、床の間、ちゃぶ台、廊下、
タイル貼り、漆喰、市場、雨戸、雨樋、衣文かけ、鴨居、長押、
ゆけむり、いとま、表札、畳庵、露天茶屋、マッチ、けむり、
赤ポスト、赤ちょうちん、木陰、受令波、傍受、Pチャン、
煙突、鉄塔、四畳半、ひなた、車両基地、浜焼き、
・春:さくら、ランドセル、
・夏:花火、夕焼け、とんぼ、海、お墓、水着、
・秋:どんぐり、紅葉、ススキ、コスモス、古本、
・冬:雪、てぶくろ、
\section{ダゲレオ暗箱の調整法}
\section{多摩川渡し碑朗読部}
・APS、arXiv朗読
 abstruct x10、paper x1
\section{ワフ1形}
\section{模型屋さん}
\section{トワイライトゾーン多摩}
\section{彫刻室、羅針盤、アンドロメダ}
・夏日星、明星、水鏡、天道
・M31、M4、M13、M7、M45、M42、M35、(NGC869\&NGC884)
・お月見 ティコ、グルマルディ、コペルニクス、アリスタルコス、ケプラー
・4月
バナナ
効能:
・5月
いちご、
効能:
そら豆
・6月
びわ、さくらんぼ
・7月
桃、アイス
・8月
イチジク、かき氷
・9月
梨、くり
・10月:朝五時、オリオン座
柿
・11月:朝四時、オリオン座
リンゴ
・12月
ケーキ
・1月
みかん
・2月
デコポン
・3月
キウイ
天文手帳 11月1日
理科年表 11月21日
\chapter{道端の物理}
\section{グリーン関数}
\section{計算尺(回転数、取り合い、落差、平均、三尺勾配、所要量、開平、開立、ねずみ算、俵杉算、方陣、摂動スペクトル、馬乗り算、油分け算)}
\section{Hamilton-Jacobi equation}
・おもりと穴の開いた円盤
・流体力学
 Navier-Stokes equation→ランキン渦
・電磁気学
・熱・統計力学
・量子力学
\section{無次元数、バッキンガムのπ定理}
アルキメデス数
フルード数
グラスホフ数
レイノルズ数
ロズビー数
ストローハル数
クヌーセン数
マッハ数
ヌセルト数
プラントル数
レイリー数
シュミット数
管摩擦係数
抗力係数
体積弾性係数
動粘性係数
摩擦抵抗係数
メルセンヌ数
\section{四畳半の道端}
\chapter{メモ}

\begin{flushright}
\LaTeX \\
\end{flushright}
\end{document}
%eof
